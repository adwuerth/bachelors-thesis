\chapter{Abstract}

We present a framework for students to bootstrap your final thesis. The primary goal of this template is to improve the quality of the thesis by avoiding typical mistakes. Our framework focuses on the basic thesis structure, which is mostly applicable, and helps you to immediately start writing. The first step is that the student writes down what he did so far, and performs some changes to this structure, yielding a thesis with notes and a rough plan how to write it up. The second step is to transform your notes into sentences, yielding a first draft. A template is the first step to the final thesis. In contrast to writing a thesis from scratch, our approach gives you a scaffold and helps you focusing on the important parts. We also show how to plot your data and describe your experiments. We present experimental results showing the perfect final thesis in the end.

As you might already have noticed, the abstract is the first, and sometimes only part the reader notices.
Thus it's crucial to summarize your work while motivating the reader here.
To help you with formulating we provide a basic structure to just fill the gaps.
Afterward, you can and should reformulate it to add your personal touch to it.

[...] present [...] for [...] to [...]. The primary goal of [...] is to improve [...] by avoiding [...] . Our framework focuses on [...], which is [...] , and is [...]. The first [...] , and performs [...], yielding [...]. The second [...] where [...], and yields [...]. [...] is the first [...]. In contrast [...], our approach [...]. We also show how [...]. We present experimental results showing [...].
 

\chapter{Kurzfassung}

In case you are not speaking german, you are free to drop the german abstract.

Dies ist der einzige Teil der Arbeit auf deutsch. 
Die Kurzfassung ist optional.
Sie erfüllt den gleichen Teil wie der englische Abstract, ist aber wahrscheinlich ein bisschen verständlicher, wenn ihr eure Arbeit daheim zeigt.