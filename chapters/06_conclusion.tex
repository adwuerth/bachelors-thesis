\chapter{Conclusion}
\paragraph{IOMMU}
In this thesis, we improved vroom's safety by implementing IOMMU support and come to the same conclusion as SPDK. The advantages of using the VFIO such as access rights and bigger address spaces as well as the ability to run the driver without root privileges overweigh the small performance impact that can be registered in niche cases. Considering that IOMMU technology has seen a rise in popularity in the use of hardware passtrough for virtualization it is also likely that in the future the IOMMU performance and the IOTLB size will increase, further closing the gap. The ability to improve security drastically and increasing address space, while not compromising on performance are the reason the MMU succeeded, and it is likely that the IOMMU will as well.

\paragraph{Rust in driver development}
The viability of using Rust to develop drivers has been shown oftentimes, and it has proved that a modern, memory-safe language like Rust can compete with C in systems development. Using Rust does not only provide more safety, but also a modern ecosystem, a package manager and zero cost abstractions. Using Rust for drivers ensures in-process memory safety.

\paragraph{Future Work}
Future Work on the driver could include expanding the NVMe capabilities. Currently the driver is fixed to one namespace. Furthermore, the driver does not support a block device layer and a file system. Including sysfs support could also be a next step.
To further push the throughput it could be investigated if and how many threads could operate on one I/O queue.