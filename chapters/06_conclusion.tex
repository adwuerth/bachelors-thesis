\chapter{Conclusion}
In this thesis, we improved Vroom's safety by implementing IOMMU support and came to the same conclusion as SPDK.
% In our tests, we only saw a noticeable performance impact on the older and slower Intel Xeon CPU, which makes us conclude that the IOMMU performance is becoming less of a factor.
We figured out the IOTLB size of two IOMMU models, which, when exceeded, can introduce address translation overhead for \qty{4}{\kibi\byte} pages.
We solved this issue using \qty{2}{\mebi\byte} pages, enabling the IOMMU to perform the same as the physical addressing.
The advantages of using the IOMMU, such as access rights and bigger address spaces, as well as the ability to run the driver without root privileges with VFIO, make it the preferred method of addressing memory.
Considering that IOMMU technology has seen a rise in popularity in the use of hardware passthrough for virtualization, it is likely that in the future, the IOMMU performance and the IOTLB size will increase, further closing any existing gap. The ability to improve security drastically and increase address space while not compromising on performance is the reason the MMU succeeded, and it is likely that the IOMMU will as well.

\paragraph{Rust in driver development}
The viability of using Rust to develop drivers has been shown often, and it has proved that a modern, memory-safe language like Rust can compete with C in systems development. Using Rust provides more safety and a modern ecosystem, a package manager, and zero-cost abstractions. This makes Rust an excellent choice for driver development.

\paragraph{Future Work}
Future Work on the driver could include expanding the NVMe capabilities. Currently, the driver is fixed to one namespace. Furthermore, the driver does not support a block device layer or file system. Including sysfs support could also be a next step.
Also, it could be investigated if and how many threads could operate on one I/O queue to further push the throughput. Finally, a performance investigation into IOMMUFD could be conducted, as we only tested the legacy VFIO model.