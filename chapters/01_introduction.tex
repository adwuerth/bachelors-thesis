\chapter{Introduction}\label{chapter:introduction}

First, a disclaimer: This document is only a template and an advice. It is not binding!
You are completely free to ignore all the advice given (if you have a reason)\footnote{This holds for all sections.}.

The introduction motivates the reader and shows the importance of the topic.
Thus it does not have to be as technical as the remainder of the thesis.
A common method it to briefly outline the development of the topic with a time frame.
One figure with some time data or other plot helps to catch the attention more easily.
It is more general and motivates that a topic in this field is interesting to consider.

This chapter contains the following:
\begin{itemize}
  \item Show importance of the topic (e.g. with time frame)
  \item Motivate the topic (e.g. by a knowledge gap or a controversy)
  \item State the focus and aim of the thesis.
  \item Explain some keyword and show the related work if its short
  \item Outline the structure of the remainder
\end{itemize}

\section{Motivation}\label{chapter:motivation}
This section motivates why you did the thesis.
For example there was a knowledge gap before or the sources cannot agree on one opinion.
Afterward, it has to be clear why and what you are doing in the thesis.
This gives the reader an impression of what to expect in this thesis.

In the end you may anticipate the end of the thesis and show a brief before, after comparison -- maybe also using a figure.
However, keep in mind that the reader may not know all technicalities yet. So it has to be as precise as possible without overusing terminologies.

\section{State of the Art}\label{chapter:stateoftheart}

This section is optional depending on how much related work you have to cover.
If it's little or very much this section may be helpful.

\paragraph{Little related work} You can describe all your related work here and give an overview how the topic evolved.
You should end in the current state of art and give the reader some insights in the field to compare your work to the work already done by others.

\paragraph{More related work} This section gives a really brief comparing summary of the related work.
It should only consist of a comparison of the most important approaches and the most important developments the reader needs to understand the topic.
You should finally forward reference to your related work section.

For both cases its beneficial to introduce terminology here and introduce the reader into your subject.

\section{Structure}\label{c:i:s}

Here you can give an outline over the remainder of the thesis to give a jump table for the reader to see which are the most interesting chapters to read.

\cite{networkx}
