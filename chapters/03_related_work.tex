\chapter{Related Work}

\section{Ixy}
Ixy is a network interface card (NIC) driver for Intel's 82599 10GbE NICs (ixgbe family) \cite{emmerich2019user}.
Ixy has been implemented in many languages, e.g., C, Go, and Rust. Ixy.rs is the Rust implementation of the Ixy driver \cite{baellmann}.
Stefan Huber implemented IOMMU support for the Ixy.rs driver. It was concluded that while using the IOMMU with \qty{2}{\mebi\byte} pages, the performance matches the performance without the IOMMU. On the other hand, it was found that using \qty{4}{\kibi\byte} pages can lead to a potential 75\% performance decrease. Additionally, Huber found that the tested Intel Xeon E5-2620 v3 6-core 2.4GHz CPU IOMMU TLB has a maximum size of 64 entries \cite{iommuhuber}. Rolf Neugebauer et al. determined the same 64 IOTLB size on the tested Intel Xeon E5-2630v4 2.2GHz \cite{pcieperfnegebauer}.

\section{Data Plane Development Kit}
The Data Plane Development Kit (DPDK) is a framework for developing userspace network card drivers. It allows for high-performance network applications. It can run using direct memory access with physical addresses or with VFIO \cite{aboutdpdk}. DPDK offers polling drivers for a variety of network cards.
It is one of the most successful projects in the world of userspace drivers and has influenced many advances in the IOMMU space.

\section{Storage Performance Development Kit}
The demand for high-speed userspace drivers in storage applications inspired the development of the Storage Performance Development Kit (SPDK). SPDK uses some shared libraries and architecture with DPDK. Primarily through the wide adoption of the NVMe protocol and the standardization of said protocol, only one driver for all NVMe SSDs has to be developed.
NVMe is a storage protocol that is widely used, modern, and highly performant. Therefore, it is a protocol for which many drivers, including userspace drivers, have been written. The Storage Performance Development Kit (SPDK) provides ``a collection of tools and libraries for writing high performance, scalable, user-mode storage applications'' \cite{spdkindex}. It includes a userspace NVMe driver, which is fast and production-ready. While this driver supports using the driver without the IOMMU, the SPDK Documentation recommends using the IOMMU as it is the "future proof...long-term foundation" for SPDK \cite{spdkmemory}. Even though SPDK is the established userspace NVMe driver option, the drawbacks include its high complexity, even for simple applications, as well as it being written in C.
